\begin{abstract}
    The three-dimensional velocities of stars can reveal their orbital
    properties and their kinematic ages.
Precise Gaia measurements of positions, parallaxes and proper motions provide
    an opportunity to measure the two-dimensional velocities of billions of
    Galactic stars.
Radial velocity (RV) measurements from spectroscopic surveys provide an
    opportunity to calculate three-dimensional velocities for some of these
    stars.
Gaia will provide RVs for stars as faint as 15th magnitude in its third data
    release, however there are now and will remain many stars without RV
    measurements.
Without RV measurements, it is still possible to infer full three-dimensional
    velocities by marginalizing over the missing RV dimension.
In this paper, we infer the three-dimensional velocities of stars in the
    Kepler field in Galactocentric coordinates (\vx, \vy, \vz).
Lying at a low galactic latitude, the Kepler field is oriented close to the
    y-axis of the Galactocentric coordinate system.
Without RVs, this means that \vy\ is poorly constrained, but \vx\ and \vy\ can
    still be precisely inferred.
For many applications, including kinematic age-dating, having precise
    velocities in the \vz\ and \vx\ directions are sufficiently useful.
This method, applied in different parts of the sky, will result in a different
    mix of precision distributed across the \vx, \vy, and \vz\ dimensions.
However, although our methodology is broadly applicable to targets across the
    sky, our prior is specifically constructed from and for the Kepler field.
Care should be taken therefore, when applying this method to other parts of
    the Galaxy, that a suitable prior is used.
\end{abstract}

