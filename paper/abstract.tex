\begin{abstract}
Precise Gaia measurements of positions, parallaxes, and proper motions provide
    an opportunity to measure the two-dimensional velocities of Milky Way
   stars.
Where available, spectroscopic radial velocity (RV) measurements provide an
    opportunity to calculate {\it three}-dimensional stellar velocities.
    % , which
    % could reveal the orbital properties and kinematic ages of Galactic stellar
    % populations.
Gaia will provide RVs for stars as faint as the 15th magnitude in its third
    data release, however there are now and will remain many stars without RV
    measurements.
Without an RV it is not possible to directly calculate 3D stellar velocities,
    however it is still possible to {\it infer} full three-dimensional stellar
    velocities by marginalizing over the missing RV dimension.
In this paper, we calculate and infer the three-dimensional velocities of
    stars in the Kepler field in Galactocentric coordinates (\vx, \vy, \vz).
Where available, we use RV measurements from the Gaia, LAMOST and APOGEE
    spectroscopic surveys, and otherwise marginalize over missing RV
    measurements.
Lying at a low Galactic latitude, the Kepler field is oriented close to the
    y-axis of the Galactocentric coordinate system.
This means that, without an RV, \vy\ is poorly constrained but \vx\ and \vy\
    can be precisely inferred.
The median uncertainties on our inferred \vx, \vy, and \vz\ velocities are
    around \vxprecision, \vyprecision, and \vzprecision\ kms$^{-1}$,
    respectively.
For many applications, including kinematic age-dating, precise velocities in
    the \vz\ and \vx\ directions are sufficiently useful.
We provide a total of 3D velocities for \nstars\ stars in the Kepler field,
    with and without RV measurements.
% This method, applied in different parts of the sky, will result in a different
%     mix of precision distributed across the \vx, \vy, and \vz\ dimensions.
% However,
Although the methodology used here is broadly applicable to targets across the
    sky, our prior is specifically constructed from and for the Kepler field.
Care should be taken to use a suitable prior when applying this method to
    other parts of the Galaxy.
\end{abstract}

