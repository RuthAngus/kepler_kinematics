\begin{abstract}
Precise Gaia measurements of positions, parallaxes, and proper motions provide
    an opportunity to calculate 3D positions and 2D velocities (\ie\
    5D phase-space) of Milky Way stars.
    % measure the two-dimensional velocities of Milky Way
   % stars.
Where available, spectroscopic radial velocity (RV) measurements provide full
    6D phase-space information.
    % , with {\it three}-dimensional stellar
    % velocities.
    % , which
    % could reveal the orbital properties and kinematic ages of Galactic stellar
    % populations.
Gaia will provide RVs for stars as faint as the 15th magnitude in its third
    data release, however there are now and will remain many stars without RV
    measurements.
Without an RV it is not possible to directly calculate 3D stellar velocities,
    however it is still possible to {\it infer} 3D stellar velocities by
    marginalizing over the missing RV dimension.
In this paper, we infer the 3D velocities of stars in the
    Kepler field in Cartesian Galactocentric coordinates (\vx, \vy, \vz).
    We directly-calculate velocities for around a third of all Kepler targets,
    using RV measurements available from the Gaia, LAMOST and APOGEE
    spectroscopic surveys.
Using the velocity distributions of these stars as our prior, we infer
    velocities for the remaining two-thirds of the sample by marginalizing
    over the RV dimension.
The Kepler field lies at a low Galactic latitude and is closely aligned with
    the y-axis of the Cartesian Galactocentric coordinate system.
This means that, without an RV, \vy\ is poorly constrained but \vx\ and \vy\
    can be precisely inferred.
The median uncertainties on our inferred \vx, \vy, and \vz\ velocities are
    around \vxprecision, \vyprecision, and \vzprecision\ kms$^{-1}$,
    respectively.
For many applications, including kinematic age-dating, precise velocities in
    the \vz\ and \vx\ directions are sufficiently useful.
We provide a total of 3D velocities for \nstars\ stars in the Kepler field,
    with and without RV measurements.
% This method, applied in different parts of the sky, will result in a different
%     mix of precision distributed across the \vx, \vy, and \vz\ dimensions.
% However,
Although the methodology used here is broadly applicable to targets across the
    sky, our prior is specifically constructed from and for the Kepler field.
Care should be taken to use a suitable prior when extending this method to
    other parts of the Galaxy.
\end{abstract}

