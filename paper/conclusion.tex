\section{Conclusion}

This paper describes a method for inferring the 3D velocities of stars by
marginalizing over missing radial velocity measurements.
We focused on stars in the Kepler field because of its potential for studying
stellar evolution via kinematic age-dating as well as its advantageous
orientation.
Located at low Galactic latitude, the Kepler field is almost aligned with the
$y$-axis of the Galactocentric coordinate system.
This means that 2D Gaia proper motion measurements alone are sufficient to
tightly constrain the \vx\ and \vz\ velocities of Kepler stars.
Without RV measurements, the \vy\ velocities  of Kepler stars are
poorly constrained.
However, given that many age-velocity dispersion relations (AVR) are
calibrated in {\it vertical} velocity, \vz\ is the main parameter of interest
for kinematic age-dating and it is precisely constrained by our method: \vz\
is inferred with a median precision of \vzprecision\ kms$^{-1}$.

We compiled kinematic data for Kepler targets from the, Gaia EDR3, LAMOST DR5
and APOGEE DR16 catalogs.
Gaia EDR3 provided parallaxes, positions and proper motions for the stars in
our sample.
Altogether, Gaia DR2, LAMOST DR5, and APOGEE DR16 provided RVs for \nrv\
Kepler targets.
% Of the three spectroscopic surveys, APOGEE has the highest resolution,
% followed by Gaia, then LAMOST, so we adopted RVs in that priority-order
% where stars had multiple RV measurements available.

We calculated \vx, \vy, and \vz\ for the \nrv\ stars in our sample with RVs
using {\tt astropy}.
For the remaining stars, we {\it inferred} \vx, \vy, \vz, and distance while
marginalizing over RV.
Our prior was a 4D Gaussian in \vx, \vy, \vz\ and $\ln$(distance), which was
based on the distribution of stars in our sample {\it with} RVs.
Since the populations of stars with and without RVs in the Kepler field are
somewhat different -- stars {\it with} RVs are generally brighter than stars
without -- we tested the sensitivity of our results to the prior.
We split the subsample of stars with measured RVs into two further subgroups:
stars brighter and stars fainter than 13th magnitude in Gaia $G$-band (13th
being the median magnitude of the Kepler stars with RVs).
Priors were constructed from the faint and bright halves of the sample and
used to infer the velocities of 1000 stars randomly selected from the total RV
sample.
Upon examination, we found the final inferred velocities were similar,
irrespective of the prior.
As expected, \vx\ and \vz\ depend very little on the prior but \vy\ has a
stronger prior-dependence because it is difficult to constrain without an RV
for Kepler stars.
A caveat of our inferred velocities is therefore that the \vy\ velocities may
not be accurate for faint stars in the Kepler field.

Overall, we found that the distribution of inferred velocities for Kepler
targets is similar to the velocity distribution of stars with measured RVs.
The median precision of inferred \vx, \vy, and \vz\ velocities is
\vxprecision, \vyprecision, and \vzprecision\ kms$^{-1}$ respectively.
We provide a table of parameters \vx, \vy, \vz, and $\ln$(distance), with
uncertainties and covariances, for a total of \nstars\ Kepler targets.
This table also contains the positional and velocity information from Gaia
DR2, Gaia EDR3, LAMOST DR5, and APOGEE DR16 used in this project.
