\section{The Data}
\label{sec:data}

We used the Kepler-Gaia cross-matched catalog available at gaia-kepler.fun,
which includes 194764 Kepler targets, cross-matched with Gaia targets within
in a 1'' radius.
% Using this catalog as a basis, we updated the Gaia parameters to those from
% EDR3.
This catalog includes Gaia positions, parallaxes, and proper distances from
Gaia EDR3 and RVs from Gaia DR2.
It also includes distances inferred from Gaia EDR3 parallaxes
\citep{bailer-jones2020}.
% created using the cross-match service and Vizier catalogue access tool
% provided by CDS, Strasbourg, France, as well as the astroquery and astropy
% python packages.
We crossmatched this catalog with the LAMOST DR5 catalog, also using a 1''
radius and, where available, added APOGEE RVs from the DR16 stellar catalog
\citep{citations}.
We removed stars with a Gaia parallax $<$ 0, parallax signal-to-noise ratio
$<$ 10, or Gaia astrometric excess noise $>$ 5.
After applying these cuts our total number of targets was 178,000 stars:
28,112 with RVs from Gaia DR2, 37,567 from LAMOST DR5, and XXX from APOGEE
DR16.
In total, 58,397 stars in our sample have RVs from {\it either} Gaia,
LAMOST, or APOGEE, 12,282 have RVs from two sources, and XXX from all three.
The APOGEE survey has a higher spectral resolution than Gaia, which is higher
than LAMOST.
The median RV uncertainty for stars in our sample is around 0.1 km/s for
APOGEE RVs, 1 km/s for Gaia RVS, and 4 km/s for LAMOST RVs.
In cases where stars had two or more available RV measurements, we adopted
APOGEE RVs as a first priority, followed by Gaia, then LAMOST.

Although RVs are available for more than one in three stars in this \kepler\
sample, most stars with RVs are bright.
Very few of the coolest stars have RV measurements because of the selection
functions of spectroscopic surveys.
% In our sample, one in 2.5 stars hotter than 5000 K had RV measurements,
% whereas only one in six stars cooler than 5000 K had RVs.
\gaia\ DR2 only includes RVs for stars brighter than around 13th magnitude,
and \lamost\ only provides RVs for \kepler\ stars brighter than around 17th
magnitude in \gaia\ $G$-band.
\racomment{Ruth, check the actual LAMOST selection function.}
Figure \ref{fig:rv_histogram} shows the apparent magnitude and temperature
distributions of the stars in our sample, with and without RVs.
This figure reveals the combined selection functions of the Gaia, LAMOST and
APOGEE RV surveys and shows that faint stars have fewer RV measurements than
bright ones.
\begin{figure}[ht!]
\caption{
    The apparent magnitude (left) and temperature (right) distributions of
    stars in our sample, with and without RV measurements from \gaia\ and
    \lamost.
}
  \centering \includegraphics[width=1\textwidth]{rv_histogram}
\label{fig:rv_histogram}
\end{figure}
Given that rotational evolution is particularly poorly understood for M
dwarfs, the cool stars with missing RVs are arguably the most interesting.
To fill-in the low-temperature regime, we inferred velocities for stars
without RV measurements, by marginalizing over missing RVs.
