\section{Introduction}

Gaia has revolutionized the field of Galactic dynamics by providing precise
positions and proper motions for an unprecendented number of stars in the
Milky Way.
So far, Gaia has provided positions and proper motions for around 1.7 billion
stars, and radial velocities (RVs) for more than 7 million stars across its
1st, 2nd and early-3rd data releases \citep{gaia, gaia_dr2}.
In combination, proper motion, position, and RV measurements provide the full
3-dimensional velocity vector for any given star, which can be used to
calculate its Galactic orbit.
% 3D stellar velocities are useful for calculating the relative space-motions of
% stars, and for revealing their Galactic orbits.
Amongst other applications, the orbits of stars can be used to explore the
secular dynamical evolution of the Galaxy, for example by studying vertical
heating and radial migration \citep{citations}, of for differentiating
nascent and accreted stellar populations in the Milky Way's halo
\citep{citation}.

RV measurements, combined with proper motions measured in the plane of the
sky, complete the information needed to calculate 3D stellar velocities.
However, RV can only be measured from a stellar spectrum -- an observation
requiring more photons than photometric or astrometric observations.
RVs are therefore difficult to obtain for faint stars.
Fortunately however, Gaia proper motion measurements are of such incredible
precision that, even without an RV measurement, the 3D velocity of a star can
still be inferred by marginalizing over radial velocity.
This will often provide a velocity that is not equally well-constrained in
every direction, \ie\ the probability density function of a star's velocity
will be an oblate spheroid in 3D.
In the equatorial coordinate system, a star's velocity will be tightly
constrained the RA and dec directions, and only constrained by the prior in
the radial direction.
Transforming to any other coordinate system, a star's velocity probability
density function will change shape via a transformation that depends on its
position.

There are several applications for which the 3D velocity of a star is useful,
even if its velocity is not equally well-constrained in every direction.
For example, \citet{oh2017} used Gaia proper motions to identify comoving
pairs and groups of stars by marginalizing over missing RVs.
In their study, the relative space motions of pairs of stars were used to
establish whether they qualified as `comoving'.
In a pathological case where two stars have near identical proper motions and
completely different RVs, their method would incorrectly flag them as comoving
stars, however in general the Gaia proper motion precision is sufficiently
high to make these cases rare.

In some cases, the velocities of stars in particular directions are more
useful than others, for example, the {\it vertical} velocities of stars (\vz\
or $W$) are often used to study the secular orbital heating of stars in the
Milky Way's disk \citep[\eg][]{ting2019, beane2018, yu2018, citations}.
Unless the radial velocity direction precisely coincides with the vertical
axis of the Milky Way, \ie\ a star lies along the $Z$ axis of the
Galactocentric coordinate system, we can still extract some \vz\ information
from Gaia proper motions alone.
Of course, the lower the Galactic latitude of a star, the better the
constraint on its \vz\ will be (hence why the Kepler field, located at low
latitude, is particularly useful for vertical velocity studies).

% Motivation
One of our main purposes in calculating the velocities of Kepler stars is to
use Galactic kinematics to date them.
The ages of stars, particularly GKM stars on the main sequence, are difficult
to measure because their luminosities and temperatures evolve slowly
\citep[see][for a review of stellar ages]{soderblom2010}.
Empirical models that relate the magnetic activity or rotation periods of
stars to their age can be used to infer the ages of some low-mass dwarfs
\citep{citations}, however, these empirical relations are often poorly
calibrated for K and M dwarfs and old stars \citep{citations}.
Galactic kinematics provides an alternative, statistical dating method.

The star forming molecular gas clouds observed in the Milky Way have a low
out-of-plane, or vertical, velocity \citep[\eg][]{stark1989, stark2005,
aumer2009, martig2014, aumer2016}.
In contrast, the vertical velocities of older stars are observed to be larger
in magnitude on average \citep{stromberg1946, wielen1977, nordstrom2004,
holmberg2007, holmberg2009, aumer2009, casagrande2011, ting2019, yu2018}.
There are two possible explanations for this observed increase in velocity
dispersion with age: either stars are born kinematically `cool' and their
orbits are heated over time via interactions with giant molecular clouds
\citep[see][for a review of secular evolution in the MW]{sellwood2014}, or
stars formed kinematically `hotter' in the past \citep[\eg][]{bird2013}.
Either way, the vertical velocity dispersions of thin disk stars are observed
to increase with stellar age.
This behavior is codified by Age-Velocity dispersion Relations (AVRs), which
typically express the relationship between age and velocity dispersion as a
power law: $\sigma_v \propto t^\beta$, with free parameter, $\beta$
\citep[\eg][]{holmberg2009, yu2018}.
These expressions can be used to infer the ages of groups of stars from their
velocity dispersions.
However, AVRs are usually calibrated in 3D Galactocentric velocities -- most
commonly in \vz\ or W.
Regardless of the coordinate system, some transformation from proper motion in
RA and dec is currently required to calculate the kinematic ages of stars.

AVRs are usually calibrated in Galactocentric velocity coordinates (\vx, \vy,
\vz\ or $UVW$), and these velocities can only be calculated with full 6D
positional and velocity information, however most Kepler stars do not
have RV measurements\footnote{Although RVs for most will be released in \gaia\
DR3}.
% In \citep{angus2020} we used velocity in the direction of Galactic latitude
% (\vb) as a stand-in for \vz\ because, in the {\it Galactic} coordinate system,
% velocities can be calculated from 3D positions and {\it 2D} proper motions.
% The \kepler\ field lies at low Galactic latitude, so \vb\ is a close
% approximation to \vz\ for Kepler stars.
% Although \vb\ velocity dispersion does not equal \vz\ velocity dispersion, it
% still increases monotonically over time and provides accurate age rankings for
% \kepler\ stars.
% In this paper however, we calculate or infer \vz\ for each Kepler star in our
% sample.
% For stars without RV measurements we infer {\it vertical} velocity, \vz, by
% marginalizing over missing RVs using a hierarchical Bayesian model.
% The vertical velocities presented in this paper were used by \citet{lu2021} to
% calculate the velocity dispersions of groups of stars with similar properties,
% which they then converted into ages using an AVR \citep{yu2018}.
% In this work, we extend the work of \citet{lu2021} by applying their kinematic
% dating method to stars with marginal rotation period detections in the Kepler
% field \citep{mcquillan2014}.
% % We compare the kinematic ages of stars with and without rotation period
% % detections, and those with marginal detections.
In \citet{angus2020} we used the velocities of Kepler stars in the direction
of Galactic latitude, \vb\, as a proxy for vertical velocity, \vz\ (because
the Kepler field lies at low Galactic latitude \vb\ is similar to \vz).
We used the velocity dispersions of stars as an age proxy, to explore the
evolution of stellar rotation rates.
In \citet{lu2021} we used vertical velocity dispersion to calculate kinematic
ages for Kepler stars with measured rotation periods using an AVR.
Those vertical velocities were inferred by marginalizing over RVs using the
method we describe in this paper.

% Kinematic ages have been used to explore the evolution of cool dwarfs for over
% a decade.
% For example, \citet{west2004, west2006} found that the fraction of
% magnetically active M dwarfs decreases over time, by using the vertical
% distances of stars from the Galactic mid-plane as an age proxy, and
% \citet{west2008} used kinematic ages to calculate the expected activity
% lifetime for M dwarfs of different spectral types.
% \citet{faherty2009} used tangential velocities to infer the ages of M, L and T
% dwarfs, and showed that dwarfs with lower surface gravities tended to be
% kinematically younger, and \citet{kiman2019} used velocity dispersion as an
% age proxy to explore the evolution of H$\alpha$ equivalent width (a magnetic
% activity indicator), in M dwarfs.
% In \citet{angus2020} we explored the kinematic properties of Kepler field
% stars and found that the velocity dispersion of stars increase with stellar
% rotation period, as expected (because both quantities increase with age).
% In \citet{lu2021} we used the velocity dispersions of Kepler field stars to
% estimate their ages using an Age-Velocity dispersion Relation (AVR).

This paper is laid out as follows.
In section \ref{sec:data} we describe the data used in this paper.
In section \ref{sec:method} we describe how we calculate the kinematic ages of
Kepler stars from their positions, proper motions and parallaxes,
marginalizing over missing RVs.
We also justify the choice of prior probability density function (PDF).
In section \ref{sec:results} we present the 3D velocities of XXX Kepler stars
and explore the accuracy and precision of our method.
