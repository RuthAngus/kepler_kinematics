\section{Introduction}

Gaia has revolutionized the field of Galactic dynamics by providing positions,
parallaxes and proper motions with unparalleled precision, for a large number
of Milky Way stars.
So far, Gaia has provided positions, parallaxes and proper motions for around
1.7 billion stars, and radial velocities (RVs) for more than 7 million stars
across its 1st, 2nd and early-3rd data releases \citep{gaia, gaia_dr2,
gaia_edr3}.
In combination, proper motion, position, and RV measurements provide full 6D
phase-space information for any given star, which can be used to calculate its
Galactic orbit.
% 3D stellar velocities are useful for calculating the relative space-motions of
% stars, and for revealing their Galactic orbits.
The orbits of stars are useful for kinematic age-dating, for exploring the
secular dynamical evolution of the Galaxy, for differentiating between nascent
and accreted stellar populations in the Milky Way's halo, and many other
applications.

% Motivation
Our main motivation for calculating the velocities of Kepler targets is to use
Galactic kinematics to study stellar evolution, either by using vertical
velocity dispersion as an age proxy, or by calculating stellar ages via an
age-velocity dispersion relation \citep[\eg][]{angus2020, lu2021}.
The ages of stars, particularly GKM stars on the main sequence, are difficult
to measure because their luminosities and temperatures evolve slowly
\citep[see][for a review of stellar ages]{soderblom2010}.
Galactic kinematics provides an alternative, statistical dating method.

The star forming molecular gas clouds observed in the Milky Way have a low
out-of-plane, or vertical, velocity \citep[\eg][]{stark1989, stark2005,
aumer2009, martig2014, aumer2016}.
In contrast, the vertical velocities of older stars are observed to be larger
in magnitude on average \citep{stromberg1946, wielen1977, nordstrom2004,
holmberg2007, holmberg2009, aumer2009, casagrande2011, yu2018, ting2019}.
Galaxy formation simulations indicate that stars initially form from
dynamically hot gas, which settles into a cooler thin disk over time during
the early stages of Galactic evolution \citep[\eg][]{bird2013}.
After this initial cooling phase, giant molecular clouds and Galactic spiral
arms dynamically `reheat' the orbits of stars over time.
% There are two possible explanations for this observed increase in velocity
% dispersion with age: either stars are born kinematically `cool' and their
% orbits are heated over time via interactions with giant molecular clouds
% \citep[see][for a review of secular evolution in the MW]{sellwood2014}, or
% stars formed kinematically `hotter' in the past \citep[\eg][]{bird2013}.
Regardless of the exact history of dynamical cooling and heating, older
populations of stars are observed to have larger velocity dispersions than
younger populations.
This behavior is codified by empirically-calibrated Age-Velocity dispersion
Relations (AVRs), which typically express the relationship between age and
velocity dispersion as a power law: $\sigma_v \propto t^\beta$, with free
parameter, $\beta$ \citep[\eg][]{holmberg2009, yu2018, mackereth2019}.
These expressions can be used to calculate the ages of stellar populations
from their velocity dispersions.
However, AVRs are usually calibrated in 3D Galactocentric velocities, and most
commonly in vertical velocity: \vz\ or W.
Regardless of the coordinate system, some transformation from RV and proper
motion in equatorial coordinates is usually required to calculate the
kinematic ages of stars using an AVR.

RV measurements, combined with positions, parallaxes, and proper motions
measured in the plane of the sky, complete the full set of information needed
to calculate 3D stellar velocities.
However, RV can only be measured from a stellar spectrum -- an observation
that requires a significant number of photons and is thus expensive to obtain,
particularly for faint stars.
Fortunately however, Gaia proper motion measurements are of such incredible
precision that, even without an RV measurement, the 3D velocity of a star can
still be inferred by marginalizing over radial velocity.
This will often provide a velocity that is not equally well-constrained in
every direction, \ie\ the probability density function of a star's velocity
will be an oblate spheroid in 3D.
In the equatorial coordinate system, a star's velocity will be tightly
constrained in the directions of RA and dec, and only constrained by
the prior in the radial direction.
Transforming to any other coordinate system, a star's velocity probability
density function will change shape via a transformation that depends on its
position.

% , however for kinematic age-dating vertical velocity, \vz, is most
% useful.
% % For example, as described above the {\it vertical} velocities of stars (\vz\
% % or $W$) are often used to study the secular orbital heating of stars in the
% % Milky Way's disk \citep[\eg][]{beane2018, yu2018, ting2019, mackereth2019}.
% Unless the radial velocity direction precisely coincides with the vertical
% axis of the Milky Way, \ie\ a star lies along the $Z$ axis of the
% Galactocentric coordinate system, we can still extract some \vz\ information
% from Gaia proper motions alone.
% Of course, the lower the Galactic latitude of a star, the better the
% constraint on its \vz\ will be (hence why the Kepler field, located at low
% latitude, is particularly useful for vertical velocity studies).
% In some cases, the velocities of stars in particular directions are more
% useful than others, for example, the {\it vertical} velocities of stars (\vz\
% or $W$) are often used to study the secular orbital heating of stars in the
% Milky Way's disk \citep[\eg][]{beane2018, yu2018, ting2019, mackereth2019}.
% Unless the radial velocity direction precisely coincides with the vertical
% axis of the Milky Way, \ie\ a star lies along the $Z$ axis of the
% Galactocentric coordinate system, we can still extract some \vz\ information
% from Gaia proper motions alone.
% Of course, the lower the Galactic latitude of a star, the better the
% constraint on its \vz\ will be (hence why the Kepler field, located at low
% latitude, is particularly useful for vertical velocity studies).

In this work, we provide 3D velocities in \vx, \vy, and \vz\ for Kepler
targets.
Our motivation is chiefly to calculate vertical velocities (\vz) which can
then be used to calculate the ages of stellar populations via an AVR, from
which other empirical age-dating methods can be calibrated.
For example, empirical or semi-empirical models that relate the magnetic
activity or rotation periods of stars to their age can be used to infer the
ages of some low-mass dwarfs \citep[\eg][]{skumanich1972, barnes2003,
barnes2007, mamajek2008, matt2012, angus2019, claytor2020}, however, these
empirical relations are often poorly calibrated for low-mass and old stars
\citep[\eg][]{angus2015, vansaders2016, vansaders2018, metcalfe2019,
curtis2020, spada2019, angus2020}.
In \citet{angus2020} we used the velocities of Kepler stars in the direction
of Galactic latitude, \vb\, as a proxy for vertical velocity.
\vb\ can be calculated without an RV and it is similar to \vz\ for many Kepler
stars because the Kepler field lies at low Galactic latitude.
We used the \vb\ velocity dispersions of stars as an age proxy to explore the
evolution of stellar rotation rates.
In \citet{lu2021} we used {\it vertical} velocity dispersion (\vz) to
calculate kinematic ages for Kepler stars with measured rotation periods using
an age-velocity dispersion relation (AVR).
Those vertical velocities were inferred by marginalizing over missing RVs
using the method we describe in this paper.
To expand upon that work and provide an opportunity to apply kinematic
age-dating to more stars, we here calculate the 3D velocities of {\it all}
Kepler targets.
Although we focus on the Kepler field, the methodology presented in this paper
is applicable to stars across the sky if a suitable prior is used.

There are several other applications for which the 3D velocity of a star is
useful, even if its velocity is not equally well-constrained in every
direction.
For example, \citet{oh2017} used Gaia proper motions to identify comoving
pairs and groups of stars by marginalizing over missing RVs.
In their study, the relative space motions of pairs of stars were used to
establish whether they qualified as `comoving'.
In a pathological case where two stars have near identical proper motions and
completely different RVs, their method would incorrectly flag them as comoving
stars, however in general the Gaia proper motion precision is sufficiently
high to make these cases rare.

Another work that predicts 3D velocities of Gaia targets is
\citet{dropulic2021}, in which the velocities of Gaia stars are predicted with
a neural network.
The network is trained on the velocities of stars {\it with} RVs and used to
predict the velocities of stars without.
The method we present here seeks to solve the same problem via a different
methodology: we use Bayesian inference instead of machine learning.
It is not possible to draw a direct comparison between their method and ours
because we use different coordinate systems (we use Cartesian, and they use
cylindrical coordinates), and because we focus on different populations --
they predict velocities for the entire Gaia catalog, whereas we concentrate on
a small part of the sky.
The two approaches will be useful for different scientific applications, and
it is extremely useful to have multiple approaches to solving this fundamental
problem in astronomy.
% Whereas our velocities are limited by the lack of information in the 6th
% dimension, theirs benefit from learned patterns in the data, and the precision
% of their resulting velocities are not fundamentally limited in the same way as
% ours are.
% However, there may be some science cases where a simple Bayesian approach,
% like the one we present here, is more suitable.

% AVRs are usually calibrated in Galactocentric velocity coordinates (\vx, \vy,
% \vz\ or $UVW$), and these velocities can only be calculated with full 6D
% positional and velocity information.
% Most Kepler stars do not currently have RV measurements.
% Some targets will have RVs released in Gaia's third data release, but many
% will not.
% In \citep{angus2020} we used velocity in the direction of Galactic latitude
% (\vb) as a stand-in for \vz\ because, in the {\it Galactic} coordinate system,
% velocities can be calculated from 3D positions and {\it 2D} proper motions.
% The \kepler\ field lies at low Galactic latitude, so \vb\ is a close
% approximation to \vz\ for Kepler stars.
% Although \vb\ velocity dispersion does not equal \vz\ velocity dispersion, it
% still increases monotonically over time and provides accurate age rankings for
% \kepler\ stars.
% In this paper however, we calculate or infer \vz\ for each Kepler star in our
% sample.
% For stars without RV measurements we infer {\it vertical} velocity, \vz, by
% marginalizing over missing RVs using a hierarchical Bayesian model.
% The vertical velocities presented in this paper were used by \citet{lu2021} to
% calculate the velocity dispersions of groups of stars with similar properties,
% which they then converted into ages using an AVR \citep{yu2018}.
% In this work, we extend the work of \citet{lu2021} by applying their kinematic
% dating method to stars with marginal rotation period detections in the Kepler
% field \citep{mcquillan2014}.
% % We compare the kinematic ages of stars with and without rotation period
% % detections, and those with marginal detections.

% Kinematic ages have been used to explore the evolution of cool dwarfs for over
% a decade.
% For example, \citet{west2004, west2006} found that the fraction of
% magnetically active M dwarfs decreases over time, by using the vertical
% distances of stars from the Galactic mid-plane as an age proxy, and
% \citet{west2008} used kinematic ages to calculate the expected activity
% lifetime for M dwarfs of different spectral types.
% \citet{faherty2009} used tangential velocities to infer the ages of M, L and T
% dwarfs, and showed that dwarfs with lower surface gravities tended to be
% kinematically younger, and \citet{kiman2019} used velocity dispersion as an
% age proxy to explore the evolution of H$\alpha$ equivalent width (a magnetic
% activity indicator), in M dwarfs.
% In \citet{angus2020} we explored the kinematic properties of Kepler field
% stars and found that the velocity dispersion of stars increase with stellar
% rotation period, as expected (because both quantities increase with age).
% In \citet{lu2021} we used the velocity dispersions of Kepler field stars to
% estimate their ages using an Age-Velocity dispersion Relation (AVR).

This paper is laid out as follows.
In section \ref{sec:data} we describe the data used in this paper.
In section \ref{sec:method} we describe how we calculate the kinematic ages of
Kepler stars from their positions, proper motions and parallaxes,
marginalizing over missing RVs.
We also justify the choice of prior probability density function (PDF).
In section \ref{sec:results} we present the 3D velocities of \nstars\ Kepler
stars and explore the accuracy and precision of our method.
