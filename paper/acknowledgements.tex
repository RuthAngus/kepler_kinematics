\section*{Acknowledgements}

% This work was partly developed at the 2019 KITP conference `Better stars,
% better planets'.
% Parts of this project are based on ideas explored at the Gaia sprints at the
% Flatiron Institute in New York City, 2016 and MPIA, Heidelberg, 2017.
{\bf
The authors would like to thank the anonymous referee whose helpful
suggestions significantly improved this manuscript.
}

This work made use of the gaia-kepler.fun crossmatch database created by Megan
Bedell.

Some of the data presented in this paper were obtained from the Mikulski
Archive for Space Telescopes (MAST).
STScI is operated by the Association of Universities for Research in
Astronomy, Inc., under NASA contract NAS5-26555.
Support for MAST for non-HST data is provided by the NASA Office of Space
Science via grant NNX09AF08G and by other grants and contracts.
This paper includes data collected by the Kepler mission. Funding for the
\Kepler\ mission is provided by the NASA Science Mission directorate.

This work has made use of data from the European Space Agency (ESA) mission
{\it Gaia} (\url{https://www.cosmos.esa.int/gaia}), processed by the {\it
Gaia} Data Processing and Analysis Consortium (DPAC,
\url{https://www.cosmos.esa.int/web/gaia/dpac/consortium}).
Funding for the DPAC has been provided by national institutions, in particular
the institutions participating in the {\it Gaia} Multilateral Agreement.

RA acknowledges support from Astrophysics Data Analysis Program award ADAP
\#80NSSC21K0636.

JCZ is supported by an NSF Astronomy and Astrophysics Postdoctoral Fellowship
under award AST-2001869.


\software{Astropy \citep{astropy2013, astropy2018}; Matplotlib
\citep{matplotlib}; Seaborn, \citep{seaborn}; Numpy, \citep{numpy}; Theano,
\citep{theano}; PyMC3, \citep{pymc3}; Exoplanet, \citep{exoplanet}}
